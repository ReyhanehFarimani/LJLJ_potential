\section{Introduction}
Multiple studies have used a modified version of Lennard-Jones potential to find the Boyle point of the colloids, at which the second virial coefficient between two such particular molecules becomes zero. And the dependence of their behavior on the solvent quality~\cite{Huissmann2009, Zahra, Narros2013}.
\\
Until now, no one has ever added the pair potential usually used in these studies in the Lammps package. Herein, we have presented the potential, so if interested, you can add it in the following path: $\textit{/src/EXTRA-PAIR}$.
Compile the Lammps package again; note that you should also add the extra-pair package.

% % Theta point:
% The $\theta$ point for a gioiven system is defined as the \textit{Boyle point}. 
% The point at which the seocnd virial coefficient between two such particular molecules become zero.
% For a fixed Hamiltonian it will be depend on N and f at the same time.
% The Boyle point becomes independent of the topology in the limit of $N \to \infty$.
% % Worsen the solvent quality
% When the solvent's quality worsens, the monomers' effective attraction increases.
% The associated changes in the form of and effective interactions between thermal stars are of high interest because of the possibilities that open up in steering macroscopic properties via temperature change.
% % Star polymers:
% Bishop and Clarke, employed Brownian Dynamics to examine the scaling laws for radius gyration they found,
% $R^2_g \propto N^{6/5}$ for athermal solvent,
% $R^2_g \propto N$ for $Theta$ solvent,
% $R^2_g \propto N^{2/3}$ for fully collapsed state.
% In fully agreement with Flory expenses.
% % effective interaction:
% The star-star interaction is most certainly affected by the solvent quality.
% Field-theoretical model -> Benhamou \textit{et al.}
% Effective interaction around the $\Theta$ point -> Good description of SANS data. (Likos \textit{et al.})
% The importance:
% Adsorption of the star into surfaces, reversible gelation, and dynamics of the solutions.

