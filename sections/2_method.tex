\section{Method}
We will use Molecular Dynamic Simulation to determine the static properties of a star polymer in three dimensions as a function of \textit{1. functionality}, \textit{2. solvent quality}.
For the polymers, we have used the Kramer-Grest model, though with an additive part in the excluded volume interaction potential. As proposed first by Huissmann \textit{et al.} we will use the usual interaction with a tunable parameter($\lambda$). Therefore, the excluded volume interaction would be as follows:
\begin{equation}
	U_{LJ}(\lambda) =      
	\begin{cases}
		4\epsilon[(\frac{\sigma}{r_{ij}})^{12} - (\frac{\sigma}{r_{ij}})^{6}] 
  + \epsilon(1 - \lambda)
  & (r_{ij}<2^{1/6}\sigma)
  \\
		4\epsilon\lambda[(\frac{\sigma}{r_{ij}})^{12} - (\frac{\sigma}{r_{ij}})^{6}] 
  & (r_{ij}\geq2^{1/6}\sigma)
	\end{cases}
\end{equation}
Here, $ r_{ij} = \vec{r}_i - \vec{r}_j $ denotes the separation between bead i and bead j, whose take position in $ \vec{r}_i $ and $ \vec{r}_j $ respectively. $ \epsilon $ governs the strength of the interaction.
Monomers are connected by a finitely extensive spring potential (FENE):
\begin{equation}
	U_{FENE} = -\frac{kR_0^2}{2} \ln\left[ 1 - (\frac{r_{ij}}{R_0})^2\right],
\end{equation}
where $ R_0 = 1.5 \sigma $ is the maximum extension length, and $ k = 30\epsilon/\sigma^2 $ is the spring constant. The length scale of the potentials is scaled for the core polymer interactions.
\\
We utilized the Lammps molecular package to simulate the system and followed the instructions in the Lammps documentation [\hyperlink{https://docs.lammps.org/Developer_write_pair.html}{Link}] to add the new pair potential for our purpose.
\\
To maintain the temperature we have used the Langevin Thermostat.
Therefore, we have solved the Langevin equation of motion as follows:
\begin{equation}
    m d^2_t\bm{r}_n = \bm{F}_n^{FENE} + \bm{F}_n^{WCA} -\eta d_t\bm{r}_n + \bm{W}_n(t)
    \label{eq:lang}
\end{equation}
 $\bm{W}_n(t)$ is a Gaussian white noise in the system satisfying the fluctuation-dissipation theorem.
\begin{equation}
    \left< \bm{W}_i(t) \cdot \bm{W}_j(t') \right> = \delta_{ij} \delta(t - t') 6k_BT \eta 
\end{equation}
The equation \ref{eq:lang}, is integrated using the Velocity-Verlet scheme, setting the MD time steps equal to $\delta t_{MD} = 0.002t_{LJ}$, where $t_{LJ} = \sqrt{\frac{m\sigma^2}{\epsilon}}$. 
Furthermore, as  in \cite{Huissmann2009}, we have set the temperature $k_BT = 1.2 \epsilon$.
\\ 
We have used two different values for the polymer functionality (namely $f = 10$, $f = 18$); all of the stars are having 50 monomers in each of the arms.